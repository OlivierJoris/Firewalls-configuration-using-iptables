\documentclass[a4paper, 11pt, oneside]{article}

\usepackage[utf8]{inputenc}
\usepackage[T1]{fontenc}
\usepackage[english]{babel}
\usepackage{array}
\usepackage{shortvrb}
\usepackage{listings}
\usepackage[fleqn]{amsmath}
\usepackage{amsfonts}
\usepackage{fullpage}
\usepackage{enumerate}
\usepackage{enumitem}
\usepackage{graphicx}
\usepackage{alltt}
\usepackage{indentfirst}
\usepackage{eurosym}
\usepackage{titlesec, blindtext, color}
\usepackage[table,xcdraw,dvipsnames]{xcolor}
\usepackage[unicode]{hyperref}
\usepackage{url}
\usepackage{float}
\usepackage{subcaption}
\usepackage[skip=1ex]{caption}

\titleformat*{\section}{\large\bfseries}

\lstset{
    language=bash,
    frame=single,
    rulecolor=\color{black}, % Couleur de la ligne qui forme le cadre
    numbers=left,
    numbersep=5pt,
    numberstyle=\tiny\color{black},
    basicstyle=\tt\footnotesize, 
    tabsize=4,
    extendedchars=true, 
    captionpos=b,
    texcl=true,
    showstringspaces=false,
    escapeinside={(>}{<)},
    inputencoding=utf8,
    literate=
  {á}{{\'a}}1 {é}{{\'e}}1 {í}{{\'i}}1 {ó}{{\'o}}1 {ú}{{\'u}}1
  {Á}{{\'A}}1 {É}{{\'E}}1 {Í}{{\'I}}1 {Ó}{{\'O}}1 {Ú}{{\'U}}1
  {à}{{\`a}}1 {è}{{\`e}}1 {ì}{{\`i}}1 {ò}{{\`o}}1 {ù}{{\`u}}1
  {À}{{\`A}}1 {È}{{\`E}}1 {Ì}{{\`I}}1 {Ò}{{\`O}}1 {Ù}{{\`U}}1
  {ä}{{\"a}}1 {ë}{{\"e}}1 {ï}{{\"i}}1 {ö}{{\"o}}1 {ü}{{\"u}}1
  {Ä}{{\"A}}1 {Ë}{{\"E}}1 {Ï}{{\"I}}1 {Ö}{{\"O}}1 {Ü}{{\"U}}1
  {â}{{\^a}}1 {ê}{{\^e}}1 {î}{{\^i}}1 {ô}{{\^o}}1 {û}{{\^u}}1
  {Â}{{\^A}}1 {Ê}{{\^E}}1 {Î}{{\^I}}1 {Ô}{{\^O}}1 {Û}{{\^U}}1
  {œ}{{\oe}}1 {Œ}{{\OE}}1 {æ}{{\ae}}1 {Æ}{{\AE}}1 {ß}{{\ss}}1
  {ű}{{\H{u}}}1 {Ű}{{\H{U}}}1 {ő}{{\H{o}}}1 {Ő}{{\H{O}}}1
  {ç}{{\c c}}1 {Ç}{{\c C}}1 {ø}{{\o}}1 {å}{{\r a}}1 {Å}{{\r A}}1
  {€}{{\euro}}1 {£}{{\pounds}}1 {«}{{\guillemotleft}}1
  {»}{{\guillemotright}}1 {ñ}{{\~n}}1 {Ñ}{{\~N}}1 {¿}{{?`}}1
}


% ==============================================================================
\title{INFO0045: Demo of MDGA.com}
\author{Maxime Goffart (s180521) \and Joris Olivier  (s182113)}
\date{Academic year 2021-2022}
% ==============================================================================

\begin{document}

\maketitle

\paragraph{}Since it is easier to illustrate what is allow than what is forbidden, we will show an example of each of the possible actions.

\section{DHCP}
\noindent Show that \texttt{U1} and \texttt{U2} have an IP address:
\begin{lstlisting}
U1> ifconfig
-----
U2> ifconfig
\end{lstlisting}

\section{PWEB and PDNS}
\noindent Show that \texttt{DT} can reach \texttt{PWEB}, using HTTP(S), and any other web pages thanks to \texttt{PDNS}:
\begin{lstlisting}
DT> lynx http://www.mdga.com
DT> lynx https://www.mdga.com
DT> lynx http://www.google.be
DT> lynx https://www.google.be
\end{lstlisting}

\section{LWEB, LDNS, and HTTP Proxy}
\noindent Show that \texttt{U1} can reach the local website and any other web pages thanks to \texttt{LDNS} and the \texttt{HTTP} proxy:
\begin{lstlisting}
U1> lynx http://local.mdga.com
U1> lynx http://www.google.be
U1> lynx https://www.google.be
\end{lstlisting}

\section{SSH}
\noindent \texttt{DT} can reach \texttt{HONEYPOT}, 192.168.1.0/24, and 192.168.2.0/24:
\begin{lstlisting}
DT> su - donald
DT> ssh HONEYPOT
DT> ssh U1
DT> ssh U2
\end{lstlisting}
\noindent \texttt{U1} can reach \texttt{PWEB} and \texttt{RSYNC}:
\begin{lstlisting}
U1> su - webteam
U1> ssh PWEB
U1> ssh RSYNC
\end{lstlisting}
\noindent \texttt{U3} can reach \texttt{U1}, \texttt{U2}, \texttt{HONEYPOT}, and \texttt{DT}:
\begin{lstlisting}
U3> su - vlad
U3> ssh U1
U3> ssh U2
U3> ssh HONEYPOT
U3> ssh DT
\end{lstlisting}

\section{SCP}
\noindent Webteam uses scp to transfer files to \texttt{PWEB}:
\begin{lstlisting}
U1> su - webteam
U1> vi text.txt # can add anything inside the file
U1> scp text.txt PWEB:/home/webteam/
-----
PWEB> su - webteam
PWEB> ls
PWEB> cat text.txt
\end{lstlisting}

\section{MAIL}
\noindent Mail servers should be reachable from outside the network. So, let us send an email from \texttt{DT}, using donald's address, to vlad.
\begin{lstlisting}
DT> su - donald
DT> mutt
\end{lstlisting}
Now, send an email to \textit{vlad@mdga.com}. Then, switch to \texttt{U3}:
\begin{lstlisting}
U3> su - vlad
U3> mutt
\end{lstlisting}
Now, read the email from donald. Then, respond to it. From \texttt{DT}, read the reply:
\begin{lstlisting}
DT> su - donald
DT> mutt
\end{lstlisting}

\section{RSYNC}
\noindent Vlad from \texttt{U3} should be able to backup to the \texttt{RSYNC} server:
\begin{lstlisting}
U3> su - vlad
U3> rsync -v Documents/opportunity.txt vlad@10.10.3.3::backup_vlad
# If we want to use secured RSYNC
U3> rsync -v Documents/opportunity.txt vlad@10.10.3.3:/home/vlad/
\end{lstlisting}

\section{NFS}
\noindent \texttt{HONEYPOT} is sharing the /home/sharing directory with the \texttt{NFS} server:
\begin{lstlisting}
HONEYPOT> cd /home/sharing
HONEYPOT> vi random.txt # add anything inside the file
-----
NFS> cd /home/sharing
NFS> cat random.txt
\end{lstlisting}

\section{FTP}
\noindent The webteam, from \texttt{U1}, can use FTP to copy data to \texttt{LWEB}:
\begin{lstlisting}
U1> su - webteam
U1> vi test.txt # can add anything inside the file
U1> ftp 10.10.2.2
U1> ls
U1> put test.txt
U1> ls
\end{lstlisting}

\end{document}